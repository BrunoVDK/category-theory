\documentclass[]{article}

\usepackage[top=1in, bottom=1.25in, left=0.9in, right=0.9in]{geometry}
\usepackage{titlesec}
\titleformat{\section}
  {\normalfont\Large}
  {\thesection}{1em}{}
\titleformat{\subsection}
  {\normalfont\large}
  {\thesection}{1em}{}
\usepackage{wallpaper}
\ThisURCornerWallPaper{1.0}{img/cover.png}
\usepackage{color}
\usepackage{xcolor}
\usepackage{hyperref}
\definecolor{blue}{rgb}{0.01, 0.28, 1.0}
\definecolor{lightgray}{rgb}{0.95, 0.95, 0.95}
\hypersetup{colorlinks=true, urlcolor=blue, citecolor=blue, filecolor=blue, linkcolor=blue}
\makeatletter
\renewcommand{\maketitle}{\bgroup\setlength{\parindent}{0pt}
\begin{flushleft}
	\vspace{0.5cm}
 	\huge{\@title}\vspace{0.5cm}\\
	\quad\Large Homework (Session 3)\vspace{0.5cm}\\
  	\qquad\large\textit{\@author}
\end{flushleft}\egroup
}
\makeatother
\newcommand{\fakesection}[1]{
	\section*{#1}
	\par\refstepcounter{section}
	\addcontentsline{toc}{section}{#1}
}
\newcommand{\fakesubsection}[1]{
	\subsection*{#1}
	\par\refstepcounter{subsection}
	\addcontentsline{toc}{subsection}{\protect\numberline{\thesubsection}#1}
}
\newcommand{\fakesubsubsection}[1]{
	\subsubsection*{#1}
	\par\refstepcounter{subsubsection}
	\addcontentsline{toc}{subsubsection}{#1}
}
\newcommand{\parx}{\par\noindent}
\usepackage{tikz}
\usepackage{listings}
\usepackage{amssymb}

\usepackage[cachedir=minted]{minted}

%%%%%%%%%%%%%%%%%%%%%%%%%%%%%%%%%%%
%%%%%%%%%%%%%%%%%%%%%%%%%%%%%%%%%%%
%%%%%%%%%%%%%%%%%%%%%%%%%%%%%%%%%%%

\begin{document}

\def\session{3}
\title{Category Theory for Programmers}
\author{Bruno Vandekerkhove}
\date{}
\maketitle
\vspace{0.5cm}

\tableofcontents

\fakesection{Definition of a Functor}

A definition of functors from Mac Lane's book\cite{book:maclane} : 

\begin{quote}
A \textit{functor} is a morphism of categories. In detail, for categories $C$ and $B$ a functor $T:C\rightarrow B$ with codomain $C$ and codomain $B$ consists of two suitably related functions: The \textit{object function $T$}, which assigns to each object $c$ of $C$ an object $T c$ of $B$ and the \textit{arrow function} (also written $T$) which assigns to each arrow $f:c\rightarrow c'$ of $C$ an arrow $T f: T c\rightarrow T c'$ of $B$, in such a way that 
$$T(1_c) = 1_{T c},\qquad T(g\circ f) = Tg\circ T f,$$
the latter whenever the composite $g\circ f$ is defined in $C$.
\end{quote}

\parx The similarities are obvious. A functor is a mapping of both objects and their morphisms from one category to an other such that identity and composition (`the structure') are preserved. An interesting simplification is \textit{nLab}'s \textit{`a functor preserves commuting diagrams.'}\cite{web:ncatlab}. A commuting diagram is a diagram where any directed path from a given start to a given end point leads to the same results. Both identities and compositions correspond to such diagrams.\\

\parx If we consider arrow-only categories (objects are represented by their identities) then an arrow-only definition of a functor follows.

\fakesection{Challenges}

\fakesubsection{Chapter 6}

\fakesubsubsection{Isomorphism of \texttt{Maybe a} and \texttt{Either () a}}

Isomorphism implies the existence of an invertible morphism, which in this case can be defined as follows :

\begin{minted}[xleftmargin=20pt,linenos]{haskell}
func :: Maybe a -> Either () a
func Nothing = Left ()
func (Just x) = Right x

func_inv :: Either () a -> Maybe a
func_inv (Left ()) = Nothing
func_inv (Right x) = Just x
\end{minted}

\fakesubsubsection{Circles and rectangles}

Let's start from a base implementation :

\begin{minted}[xleftmargin=20pt,linenos]{c++}
#include <iostream>
#include <math.h>

class Shape {
    virtual float area() = 0;
};

class Circle: public Shape {
    float _radius;
public:
    Circle(float radius) { _radius = radius; }
    float area() { return M_PI * _radius * _radius; }
};

class Rectangle: public Shape {
    float _width, _height;
public:
    Rectangle(float width, float height) { _width = width; _height = height; }
    float area() { return _width * _height; }
};
\end{minted}

We can add an operation to calculate the circumference :

\begin{minted}[xleftmargin=20pt,linenos]{c++}
...

class Shape {
    virtual float area() = 0;
    virtual float circ() = 0;
};

class Circle: public Shape {
...
    float circ() { return 2 * M_PI * _radius; }
};

class Rectangle: public Shape {
...
    float circ() { return 2 * (_width + _height); }
};
\end{minted}

\noindent Clearly, a new virtual function had to be added to the \texttt{Shape} class and implementations had to be coded for each subtype. Finally, let's add a new type of shape, a square :

\begin{minted}[xleftmargin=20pt,linenos]{c++}
...

class Square: public Shape {
    float _width;
public:
    Square(float width) { _width = width; }
    float area() { return _width * _width; }
    float circ() { return 4 * _width; }
};
\end{minted}

\noindent This time only some subtype \texttt{Square} had to be added and the virtual functions had to be implemented. In Haskell one could alter the definition of \texttt{Shape} and change each operation's implementation (adding a case for both \texttt{area} and \texttt{circ}).\\

\parx This is a reference to the \textit{expression problem} ; in functional programming languages it's easier to add operations to existing data types, but it's more difficult to add new data types. The opposite holds for object-oriented languages, where the visitor pattern helps to turn the problem on its head but does not solve it.

\fakesubsubsection{Show that \texttt{a + a = 2 $\times$ a}}

We can simply substitute $1 + 1$ for $2$ and use the distributivity property :
$$2\times a = (1 + 1)\times a = 1\times a + 1\times a = a + a$$
Or in Haskell :

\begin{minted}[xleftmargin=20pt,linenos]{haskell}
func :: Either a a -> (Bool, a)
func (Left x) = (True, x)
func (Right x) = (False, x)

func_inv :: (Bool, a) -> Either a a
func_inv (True, x) = Left x
func_inv (False, x) = Right x
\end{minted}

\fakesubsection{Chapter 7}

Proving the functor laws means that we have to show that \texttt{fmap} preserves identity and composition. When we ignore both arguments when lifting :
\begin{minted}[xleftmargin=20pt,linenos]{haskell}
fmap _ _ = Nothing
\end{minted}
Then the identity function is not always preserved :
\begin{minted}[xleftmargin=20pt,linenos]{haskell}
fmap id (Just x) = Nothing = id Nothing /= id (Just x)
\end{minted}
The laws do hold for the reader functor :
\begin{minted}[xleftmargin=20pt,linenos]{haskell}
fmap id f = (.) id f = id f = f
fmap (h . g) f = (h . g) . f =  h . (g . f) = h . (fmap g f) = fmap h (fmap g f)
\end{minted}
For the list functor they hold as well. We start with proving that identity is preserved :
\begin{minted}[xleftmargin=20pt,linenos]{haskell}
fmap id Nil = Nil = id Nil
fmap id (Cons x t) = Cons (id x) (fmap id t) = Cons x t
\end{minted}
The last line assumes that the law holds for the tail of the list. Composition is also preserved :
\begin{minted}[xleftmargin=20pt,linenos]{haskell}
fmap (f . g) Nil = Nil = fmap g Nil = fmap f (fmap g Nil)
fmap (f . g) (Cons x t) 
	= Cons ((f . g) x) (fmap (f . g) t) 
	= Cons ((f . g) x) (fmap f (fmap g t))
	= fmap f (Cons (g x) (fmap g t))
	= fmap f (fmap g (Cons x t))
\end{minted}

\fakesubsubsection{Implementing the \texttt{Reader} functor}

The \texttt{Reader} functor takes a mapping $a\rightarrow b$ and transforms it to a mapping $(r\rightarrow a)\rightarrow(r\rightarrow b)$. In Scala :
\begin{minted}[xleftmargin=20pt,linenos]{scala}
object Reader extends App {
  print(new Reader[String].fmap((i: Int) => 10 + i)(s => s.toInt * 2)("10")) // 30
}

class Reader[R] extends Functor[({type Func[X] = (R => X)})#Func] {
  override def fmap[A, B](f: A => B)(g: R => A): R => B = g andThen f
}

trait Functor[F[_]] {
  def fmap[A, B](f: (A => B))(g: F[A]) : F[B]
}
\end{minted}


\bibliographystyle{plain}
\bibliography{References}
\addcontentsline{toc}{section}{References}

\end{document}